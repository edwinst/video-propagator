\documentclass{beamer}

\usetheme{Berlin}

\newcommand{\envelope}{\frac{p}{\sqrt{p^2 + m^2}}}
\newcommand{\integrand}{\frac{p e^{ipr}}{\sqrt{p^2 + m^2}}}
\newcommand{\CC}{\mathbb{C}}

\title{Contour integral of free field propagator}
\author{Edwin Steiner}
\date{\today}


\begin{document}


\begin{frame}
\titlepage
\end{frame}


\begin{frame}
2-point function of a free scalar field, for purely spatial separations $r > 0$:

$$\int_{-\infty}^{\infty} dp \integrand.$$

\pause
Problem: \alert{This integral does not converge!} (Neither in Lebesgue's
nor Riemann's sense.)
\end{frame}


\begin{frame}
Let's consider the integrand

$$\integrand$$

as a complex function of $p \in \CC$.

\end{frame}


\begin{frame}
Actually, let's first look at:

$$\envelope$$

\pause
No surprises on the real line:

\pause
$$\envelope \ \rightarrow \ 1 \ \textrm{ for } p \rightarrow +\infty,$$

$$\envelope \ \rightarrow \ -1 \ \textrm{ for } p \rightarrow -\infty.$$

\end{frame}

% animate envelope on the real line, varying m
%: animate(3, sub { sprintf '--envelope --z0 -10 --z1 10 --m %g', 1 });

%: animate(10, sub {
%:     my $m = t_xaby(5, 0.1, 0.1, 0.1, 1, 0.2, 3, 1, $_[0]);
%:     [sprintf('--envelope --z0 -10 --z1 10 --m %g', $m),
%:      sprintf(qq{
%:          --pre-plot 'set label "varying the mass..." at 0.25,2 font "serif,20"'
%:          --pre-plot 'set label "m = %4.2g" at 0.25,1.5 font "serif,20"'}, $m)]
%: });


\begin{frame}
It gets more interesting for complex $p$:

$$\envelope$$

\begin{itemize}

\pause
\item For $|p| \rightarrow \infty$ this goes to  a pure phase.

\pause
\item There are branch point \alert{singularities} at $p = im$ and $p = -im$.

\pause
\item The \alert{branch cut} lies along the imaginary axis towards $+i\infty$ and $-i\infty$.
(Due to the branch cut of the square root on the negative real axis.)
\end{itemize}
\end{frame}

% animate envelope on the real line, rotate to imaginary axis

% animate to show the branch cut

%: animate(20, sub {
%:     my $t = t_ab(4, 0.1, 0.1, 0.1, $_[0]);
%:     $t = sin($t*pi/2)**2;
%:     my $y = 2*$t;
%:     my $preplot = ($y > 1.2) ? 'set label "branch cut" at 0.5,2 center font "serif,20"' : '';
%:     [sprintf('--envelope --z0 -10,%g --z1 10,%g --m 1', $y, $y),
%:      "--pre-plot '$preplot'"]
%: });

\begin{frame}
Including the exponential we get

$$\integrand$$

which is

\begin{itemize}
\pause
\item \alert{oscillating} along the real axis,

\pause
\item \alert{exponentially damped} in the upper half plane.
\end{itemize}
\end{frame}

% animate integrand on the real line, rotating to vertical segments to show damping

%: animate(3, sub { '--integrand --z0 -10 --z1 10 --m 1 --r 2' });

%: animate(12, sub {
%:     my $r = t_xaby(5, 0.1, 0.1, 0.1, 2, 0.5, 10, 2,$_[0]);
%:     [sprintf('--integrand --z0 -10 --z1 10 --m 1 --r %.3g', $r),
%:      sprintf(qq{
%:          --pre-plot 'set label "varying r..." at 0.25,2 font "serif,20"'
%:          --pre-plot 'set label "r = %4.2g" at 0.25,1.5 font "serif,20"'}, $r)]
%: });

%: animate(10, sub {
%:     my $t = t_ab(2, 0.1, 0.1, 0.1, $_[0]);
%:     $t = sin($t*pi/2)**2;
%:     my $y = 2*$t;
%:     my $preplot = ($y > 1.2) ? 'set label "branch cut" at 0.5,2 center font "serif,20"' : '';
%:     [sprintf('--integrand --z0 -10,%.3g --z1 10,%.3g --m 1', $y, $y),
%:      "--pre-plot '$preplot'"]
%: });

%: animate(16, sub {
%:     my $t = t_ab(2, 0.1, 0.2, 0.1, $_[0]);
%:     $t = sin($t*pi/2)**2;
%:     my @z0 = interpol_complex(-10,0, 0.01,0, $t);
%:     my @z1 = interpol_complex(+10,0, 0.01,2, $t);
%:     my $preplot = '';
%:     [sprintf('--integrand --z0 %.3g,%.3g --z1 %.3g,%.3g --m 1 --r 2', @z0, @z1),
%:      "--pre-plot '$preplot'"]
%: });


\begin{frame}
Let's try a sharp cutoff at $p = \pm \Lambda$:

$$\int_{-\Lambda}^{+\Lambda} dp \integrand$$

It looks like this as a function of $r$ (varying $\Lambda$):
\end{frame}

% show integral, vary Lambda, vary m(?)

%: $opts = '--n 250 --z0 0.1 --z1 5 --m 1';
%: animate(3, sub { $opts.' --Pr 45 --Pi 0 --d 0' });

%: animate(20, sub {
%:     my $Pr = t_xaby(5, 0.1, 0.1, 0.1, 45, 25, 60, 45, $_[0]);
%:     [sprintf($opts.' --Pr %.3g --Pi 0 --d 0', $Pr),
%:      sprintf(qq{
%:          --pre-plot 'set label "varying the cutoff..." at 5,8 font "serif,20"'
%:          --pre-plot 'set label "cutoff = %4.2g" at 5,5 font "serif,20"'}, $Pr)]
%: });


\begin{frame}
We see:

\begin{itemize}
\pause
\item Strong \alert{oscillation} at the cutoff (spatial) frequency

\pause
\item superposed onto a function falling \alert{exponentially} with $r$.
\end{itemize}

\pause
By deforming the integration contour we can show that indeed \alert{only}
the cutoff momentum contributes to the oscillations.

\pause
Witness the miracle of \alert{Cauchy's theorem}!
% XXX look up
\end{frame}

% animate the M contour

%: animate(3, sub { $opts.' --Pr 45 --Pi 0 --d 0' });

%: animate(10, sub {
%:     my $t = t_xaby(1, 0.1, 0.1, 0.1, 0, 45, 0, 45, $_[0]);
%:     [sprintf($opts.' --Pr 45 --Pi %.3g --d 0', $t),
%:      sprintf(qq{
%:          --pre-plot 'set label "pushing the contour..." at 5,8 font "serif,20"'})]
%: });

%: animate(10, sub {
%:     my $Pr = t_xaby(3, 0.1, 0.1, 0.1, 45, 25, 60, 45, $_[0]);
%:     [sprintf($opts.' --Pr %.3g --Pi 45 --d 0', $Pr),
%:      sprintf(qq{
%:          --pre-plot 'set label "varying the cutoff..." at 5,8 font "serif,20"'
%:          --pre-plot 'set label "cutoff = %4.2g" at 5,5 font "serif,20"'}, $Pr)]
%: });

%: animate(10, sub {
%:     my $t = t_xaby(2, 0.1, 0.1, 0.1, 0, 0.1, 1.0, 1.0, $_[0]);
%:     [sprintf($opts.' --Pr 45 --Pi 45 --d %.3g', $t),
%:      sprintf(qq{
%:          --pre-plot 'set label "moving the endpoints..." at 5,8 font "serif,20"'})]
%: });

%: animate(5, sub {
%:     my $t = t_xaby(1, 0.1, 0.1, 0.1, 1.0, 1.0, 1.0, 2.0, $_[0]);
%:     [sprintf($opts.' --Pr 45 --Pi 45 --d %.3g', $t),
%:      sprintf(qq{
%:          --pre-plot 'set label "moving the endpoints..." at 5,8 font "serif,20"'})]
%: });

%: animate(15, sub {
%:     my $t = t_xaby(4, 0.1, 0.1, 0.1, 2.0, 3.0, 2.8, 2.0, $_[0]);
%:     [sprintf($opts.' --Pr 45 --Pi 45 --d %.3g', $t),
%:      sprintf(qq{
%:          --pre-plot 'set label "moving the endpoints..." at 5,8 font "serif,20"'})]
%: });


% XXX smoothing


\begin{frame}
\frametitle{Software Credits}

Numerical integration: GNU Scientific Library

Plotting: gnuplot

Presentation: \TeX + \LaTeX + Beamer, avconv
\end{frame}
\end{document}
