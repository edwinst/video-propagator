\documentclass{beamer}

\newcommand{\envelope}{\frac{p}{\sqrt{p^2 + m^2}}}
\newcommand{\integrand}{\frac{p e^{ipr}}{\sqrt{p^2 + m^2}}}
\newcommand{\CC}{\mathbb{C}}

\begin{document}
\begin{frame}
2-point function of a free scalar field, for purely spatial separations $r > 0$:

$$\int_{-\infty}^{\infty} dp \integrand.$$

Problem: This integral does not converge! (Neither in Lebesgue's
nor Riemann's sense.)
\end{frame}


\begin{frame}
Let's consider the integrand

$$\integrand$$

as a complex function of $p \in \CC$.

\end{frame}


\begin{frame}
Actually, let's first look at

$$\envelope$$

No surprises on the real line:

$$\envelope \ \rightarrow \ 1 \ \textrm{ for } p \rightarrow +\infty,$$

$$\envelope \ \rightarrow \ -1 \ \textrm{ for } p \rightarrow -\infty.$$

\end{frame}

% animate envelope on the real line, varying m
%: animate(3, sub { sprintf '--envelope --z0 -10 --z1 10 --m %g', 1 });

%: animate(10, sub {
%:     my $m = t_xaby(5, 0.1, 0.1, 0.1, 1, 0.2, 3, 1, $_[0]);
%:     [sprintf('--envelope --z0 -10 --z1 10 --m %g', $m),
%:      sprintf(qq{
%:          --pre-plot 'set label "varying the mass..." at 0.25,2 font "serif,20"'
%:          --pre-plot 'set label "m = %4.2g" at 0.25,1.5 font "serif,20"'}, $m)]
%: });


\begin{frame}
It gets more interesting for complex $p$:

$$\envelope$$

For $|p| \rightarrow \infty$ this goes to  a pure phase.

But there are branch point singularities at $p = im$ and $p = -im$.

With the branch cut of the square root on the negative real axis,
the branch  cut of the integrand lies along the imaginary axis towards $+i\infty$ and $-i\infty$.
\end{frame}

% animate envelope on the real line, rotate to imaginary axis

% animate to show the branch cut

%: animate(24, sub {
%:     my $t = t_ab(5, 0.1, 0.1, 0.1, $_[0]);
%:     $t = sin($t*pi/2)**2;
%:     my $y = 2*$t;
%:     my $preplot = ($y > 1.2) ? 'set label "branch cut" at 0.5,2 center font "serif,20"' : '';
%:     [sprintf('--envelope --z0 -10,%g --z1 10,%g --m 1', $y, $y),
%:      "--pre-plot '$preplot'"]
%: });

\begin{frame}
Including the exponential we get

$$\integrand$$

which is

\begin{itemize}
\item \emph{oscillating} along the real axis,

\item \emph{exponentially damped} in the upper half plane.
\end{itemize}
\end{frame}

% animate integrand on the real line, rotating to vertical segments to show damping

%: animate(3, sub { '--integrand --z0 -10 --z1 10 --m 1 --r 2' });

%: animate(10, sub {
%:     my $r = t_xaby(5, 0.1, 0.1, 0.1, 2, 0.5, 10, 2,$_[0]);
%:     [sprintf('--integrand --z0 -10 --z1 10 --m 1 --r %.3g', $r),
%:      sprintf(qq{
%:          --pre-plot 'set label "varying r..." at 0.25,2 font "serif,20"'
%:          --pre-plot 'set label "r = %4.2g" at 0.25,1.5 font "serif,20"'}, $r)]
%: });

%: animate(24, sub {
%:     my $t = t_ab(5, 0.1, 0.1, 0.1, $_[0]);
%:     $t = sin($t*pi/2)**2;
%:     my $y = 2*$t;
%:     my $preplot = ($y > 1.2) ? 'set label "branch cut" at 0.5,2 center font "serif,20"' : '';
%:     [sprintf('--integrand --z0 -10,%g --z1 10,%g --m 1', $y, $y),
%:      "--pre-plot '$preplot'"]
%: });

%: animate(24, sub {
%:     my $t = t_ab(2, 0.1, 0.1, 0.1, $_[0]);
%:     $t = sin($t*pi/2)**2;
%:     my @z0 = interpol_complex(-10,0, 0.01,0, $t);
%:     my @z1 = interpol_complex(+10,0, 0.01,2, $t);
%:     my $preplot = '';
%:     [sprintf('--integrand --z0 %.3g,%.3g --z1 %.3g,%.3g --m 1 --r 2', @z0, @z1),
%:      "--pre-plot '$preplot'"]
%: });


\begin{frame}
Let's try a sharp cutoff at $p = \pm \Lambda$:

$$\int_{-\Lambda}^{+\Lambda} dp \integrand$$

It looks like this as a function of $r$ (varying $\Lambda$):
\end{frame}

% show integral, vary Lambda, vary m(?)


\begin{frame}
There's a strong oscillation at the cutoff (spatial) frequency $\Lambda$
which seems superposed onto a function falling exponentially with $r$.

By deforming the integration contour we can show that it is indeed only
the cutoff momentum that contributes to the oscillations.

Witness the miracle of Cauchy's theorem!
\end{frame}

% animate the M contour




% XXX smoothing


\begin{frame}
\frametitle{Software Credits}

Numerical integration: GNU Scientific Library

Plotting: gnuplot

Presentation: \LaTeX, avconv
\end{frame}
\end{document}
